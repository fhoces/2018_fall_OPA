
\documentclass{beamer} 


\mode<presentation>
{
  \usetheme[hideothersubsections]{Boadilla}
  % or ...

  \setbeamercovered{transparent}
  % or whatever (possibly just delete it)
}

\usepackage{tikz}
\usepackage{graphicx}
\usepackage[english]{babel}
\usepackage{tikz}
\usetikzlibrary{decorations.pathreplacing,angles,quotes}
\usetikzlibrary{shapes, shadows, arrows, positioning,fit}
\usepackage{multirow, booktabs, adjustbox}
\usepackage{relsize, lscape}
\usepackage[utf8]{inputenc}
\usepackage{adjustbox}
\usepackage{verbatim}
\usepackage{listings, amsmath}
\usepackage{natbib}
\usepackage{tcolorbox}
\usepackage{transparent}
\usepackage{comment}


\usepackage{relsize, lscape}
\usepackage{multirow, booktabs, adjustbox}
\usepackage{lipsum} % Used for inserting dummy 'Lorem ipsum' text into the template
\usepackage{array}
\usepackage{graphicx}
\usepackage{hyperref}
\usepackage{multicol}


% or whatever

\usepackage{times}
\usepackage[T1]{fontenc}
% Or whatever. Note that the encoding and the font should match. If T1
% does not look nice, try deleting the line with the fontenc.


\newcommand{\nit}[1]{\textrm{\textit{{\color{red}[#1]}}}}
\newcommand{\highlight}[1]{\colorbox{yellow}{$\displaystyle #1$}}
\newcommand{\highlightw}[1]{\colorbox{white}{$\displaystyle #1$}}
\newcommand{\highlightr}[1]{\colorbox{red}{$\displaystyle #1$}}
\newcommand{\tabitem}{~~\llap{\textbullet}~~}


\def\gray{\color{lightgray}}
\def\red{\color{red}}
\def\white{\color{white}}
\def\brown{\color{brown}}
\def\black{\color{black}}
\def\blue{\color{blue}}

\beamertemplatenavigationsymbolsempty

\renewcommand{\topfraction}{0.9}
\newsavebox{\codebox}
\lstset{
basicstyle=\scriptsize\tt,
}

\newcolumntype{P}[1]{>{\raggedright\arraybackslash}p{#1}}

\newcommand{\backupbegin}{
   \newcounter{finalframe}
   \setcounter{finalframe}{\value{framenumber}}
}
\newcommand{\backupend}{
   \setcounter{framenumber}{\value{finalframe}}
}

\newcommand\Wider[2][3em]{%
\makebox[\linewidth][c]{%
  \begin{minipage}{\dimexpr\textwidth+#1\relax}
  \raggedright#2
  \end{minipage}%
  }%
}
%control cross slide button format
\setbeamertemplate{button}{\tikz
  \node[
  inner xsep=2pt,
  draw=structure!50,
  fill=structure!50,
  rounded corners=2pt]  {\Large\insertbuttontext};}


\subject{Research Transparency}
% This is only inserted into the PDF information catalog. Can be left
% out. 

% If you have a file called "university-logo-filename.xxx", where xxx
% is a graphic format that can be processed by latex or pdflatex,
% resp., then you can add a logo as follows:

% \pgfdeclareimage[height=0.5cm]{university-logo}{university-logo-filename}
% \logo{\pgfuseimage{university-logo}}



% Delete this, if you do not want the table of contents to pop up at
% the beginning of each subsection:
%\AtBeginSubsection[]
%{
%  \begin{frame}<beamer>{Outline}
%    \tableofcontents[currentsection,currentsubsection]
%  \end{frame}
%}


% If you wish to uncover everything in a step-wise fashion, uncomment
% the following command: 

\beamerdefaultoverlayspecification{<.->}


\begin{document}

\title[] % (optional, use only with long paper titles)
{Open Policy Analysis: Principles and Applications}

\subtitle
{}

\author[] % (optional, use only with lots of authors)
{Fernando~Hoces de la Guardia}
% - Give the names in the same order as the appear in the paper.
% - Use the \inst{?} command only if the authors have different
%   affiliation.

\institute[] % (optional, but mostly needed)
{%
  UC Berkeley:\\
  Berkeley Initiative for Transparency in the Social Sciences\\
}
% - Use the \inst command only if there are several affiliations.
% - Keep it simple, no one is interested in your street address.

\date[] % (optional, should be abbreviation of conference name)
{GiveWell\\
August 23rd, 2018}



\begin{frame}
  \titlepage
\end{frame}


\setbeamercovered{invisible}
\begin{comment}

\begin{frame}{Motivation: To Producers of Policy Analysis}
\vspace{-0.5in}
\begin{align*}
\intertext{Cynical view:}
\text{Policy Analysis} &= \text{Research} - \text{Novelty} - \text{Rigor}  
\onslide<2->{\intertext{Optimists view:}
\text{Policy Analysis} &= \text{Research} - \text{Novelty}  + \text{Relevance}  - \text{Rigor} }
\onslide<3>{\intertext{Our proposal:}
\text{Open Policy Analysis} &= \text{Research} - \text{Novelty}+ \text{Relevance}    }
\end{align*}

\end{frame}


\begin{frame}{Motivation: To Producers of Research}
Think for a second of the one paper that you are most proud of. 
Now think of the best estimate that you have in that paper. 
\bigskip
\pause
\begin{itemize}
\item Do you know if that paper had any \textbf{direct} effect on public policy?
(direct: estimate used in policy report, law, testimony\\
indirect: general knowledge, NYT op-ed)
\pause
\item If that estimate where to be revise to by a factor of 2 (or 10). How should the policy analysis change? 
\end{itemize}
\end{frame}

\end{comment}

\setbeamercovered{transparent}

\begin{frame}
\begin{enumerate}
\item Why we need Open Policy Analysis (Hoces de la Guardia, Grant \& Miguel, 2018)
\bigskip
\item Application to policy estimates of the minimum wage. 
\end{enumerate}
\end{frame}


% Structuring a talk is a difficult task and the following structure
% may not be suitable. Here are some rules that apply for this
% solution: 

% - Exactly two or three sections (other than the summary).
% - At *most* three subsections per section.
% - Talk about 30s to 2min per frame. So there should be between about
%   15 and 30 frames, all told.

% - A conference audience is likely to know very little of what you
%   are going to talk about. So *simplify*!
% - In a 20min talk, getting the main ideas across is hard
%   enough. Leave out details, even if it means being less precise than
%   you think necessary.
% - If you omit details that are vital to the proof/implementation,
%   just say so once. Everybody will be happy with that.
%%%%%%%%%%%%%%%%%%%%%%%%%%%%%%%%%%%%%%%%%%%%%%%%%%%%%%%%%%%%%%%%%%%%%%%
%%%%%%%%%%%%%%%%%%%%%%%%%%%%%%%%%%%%%%%%%%%%%%%%%%%%%%%%%%%%%%%%%%%%%
 
\section[Evidence Based]{Policy Analysis And The Evidence-Based Policy Movement}

\begin{frame}{Policy Analysis And The Evidence-Based Policy Movement}
Evidence-Based movement is growing. 
\begin{itemize}
\item ``The golden age of evidence-based policy'' (Haskins 2017).
%\begin{itemize}
\item Credible causal evidence (Angrist \& Pischke, 2010)
\item Transparency and reproducibility of research (Miguel et al. 2014).
%\end{itemize} 
\item Commission on Evidence-Based Policymaking (CEBP, 2017)
\end{itemize}
\pause
Policy Analysis is a fundamental link. 
\begin{itemize}
\item As many definitions as textbooks (Dunn, 2015; Weimer \& Vining, 2017; Williams, 1971)
\item Common denominator: client-oriented empirical analysis meant to inform a specific policy debate
\item Aspires at scientific rigor. (Wildavsky 1979),
\end{itemize}
\end{frame} 

\begin{frame}{Examples of Policy Analysis}
\begin{figure}[h!]
\centering
\hspace*{-3em}
\includegraphics[scale = 0.3]{../Images/givewell}
\caption{Screen shot of GiveWell's CE spreadsheet}
\label{ex_pa}
\end{figure}	
\end{frame}
\begin{frame}{One Ideal Evidence-Based Policy Link}

\tikzstyle{agent} = [diamond, draw, node distance= 7em, minimum height=5em, minimum width=5em]
\tikzstyle{line} = [draw, -stealth]

\tikzstyle{line_enph} = [draw, red, ultra thick]
\tikzstyle{inp} = [draw, circle, text centered, minimum height=2em, text width=2em, node distance= 10em]
\tikzstyle{outp} = [draw, circle, text centered, minimum height=2em, text width=2em, node distance= 10em]
\tikzstyle{block1} = [draw, circle,  text width=5em, text centered, minimum height=7em, node distance= 5em]
\tikzstyle{block2} = [draw, circle,  text width=7em, text centered, minimum height=2em, node distance= 5em]
\tikzstyle{block3} = [draw, rounded rectangle,  text width=5em, text centered, minimum height=2em, node distance= 5em]


\begin{figure}[h!]\centering

\begin{tikzpicture}[thick,scale=0.6, every node/.style={scale=0.6}]

%% The oddly shaped truth
\node[regular polygon, regular polygon sides=3,
              draw, fill=white,
              inner sep=.1em,
              shape border rotate=70](tru){Truth};


%%%%%Researcher 1 and Inputs: depends on R_1%%%%%%%
\node [block1, right = 2em of tru](R_1){Research\linebreak $(R)$};
\node [block1,inner sep=0em, right = 2em of R_1](PA_1){Policy \linebreak Analysis: ($PA$) \linebreak  Gains  \& \linebreak losses};
\node [block1, above right = 0em and 8em of R_1](PM_1){Policy\linebreak Maker 1};
\node [block1, below right = 0em and 8em of R_1](PM_2){Policy\linebreak Maker 2};

\node [block3, right = 2em of PM_1](PC_1){Support};
\node [block3, right = 2em of PM_2](PC_2){Oppose};


%%%%% Draw edges col 2%%%%%%%%%%%%%%%%%
%Research  to PA
\draw [line](tru) -- (R_1);
\draw [line](R_1) -- (PA_1);
%PA to PM
\draw [line](PA_1) -- (PM_1);
\draw [line](PA_1) -- (PM_2);
%PM to PC
\draw [line](PM_1) -- (PC_1);
\draw [line](PM_2) -- (PC_2);

%%%%%%%%%%%%%%%%%%%%%%%%%%%%%%%%%%%%%%



\draw[decoration={brace}, decorate] (7,3.4) -- node[above=6pt](lab4){{\Large Observed by citizens} }(15.2,3.4);




\end{tikzpicture}
\vspace{-.8em}
\end{figure}
\end{frame}


\section[Crisis in Research]{Reproducibility Crisis In Empirical Research}

\begin{frame}{Reproducibility Crisis In Empirical Research}

\begin{itemize}
%\item  Researchers believe in good science but don't practices it (Anderson, Martinson, \& De Vries 2007).  
%\pause
%\item ``Most published research is false'' Ioannidis (2005). 
\item Large magnitude of publication bias (Franco et al 2014).  
\item Evidence of extensive p-hacking across social science disciplines (Gerber et al 2008, Brodeur et al 2016).
\item Replication rates are low (Collaboration et al, 2015 , Camerer et al, 2016). 
\item Computational reproducibility is also low (Stodden et al 2016, Chang and Li 2015, Gertler et al 2018).
\end{itemize}

\end{frame} 


\begin{frame}{The Open Science Movement}

\begin{itemize}
\item Definition of principles of Open Science/Research Transparency (Miguel et al 2014)
\item Development of guidelines to operationalize principles of Open Science (Nosek et al 2015)
\item Journals and funders: Journals (Science + 5k other journals), Registries (AEA), Funders (NIH, NSF and multiple donors)
\end{itemize}

\end{frame} 

\section[Crisis in PA]{Credibility Crisis Of Policy Analysis}

\begin{frame}{Credibility Crisis Of Policy Analysis}
\begin{itemize}
\item Incredible Certitudes  (Manski, 2013) 
\item Report wars (Wesselink et al, 2013) 
\pause
\item Alternative facts (``The Death of Expertise'' Nichols, 2017; ``The Death of Truth'', Kakutani 2018; ``Truth Decay'', Rich \& Kavanagh 2018)
\end{itemize}
\end{frame} 

\begin{frame}[shrink=30]{How This Affects The Evidence Based Policy Link?}
\pause

\tikzstyle{agent} = [diamond, draw, node distance= 7em, minimum height=5em, minimum width=5em]
\tikzstyle{line} = [draw, -stealth]

\tikzstyle{line_d} = [draw, dashed, -stealth]
\tikzstyle{inp} = [draw, circle, text centered, minimum height=2em, text width=2em, node distance= 10em]
\tikzstyle{outp} = [draw, circle, text centered, minimum height=2em, text width=2em, node distance= 10em]
\tikzstyle{block1} = [draw, circle,  text width=3em, text centered, minimum height=2em, node distance= 5em]
\tikzstyle{block2} = [draw, circle,  text width=4em, text centered, minimum height=2em, node distance= 5em]
\tikzstyle{block3} = [draw, rounded rectangle,  text width=5em, text centered, minimum height=2em, node distance= 5em]

\begin{figure}[h!]
\centering
\hspace*{0.2\linewidth}
\begin{tikzpicture}[thick,scale=0.2, every node/.style={scale=0.8}]

%% The oddly shaped truth
\node[regular polygon, regular polygon sides=3,
              draw, fill=white,
              inner sep=.1em,
              shape border rotate=70](tru){Truth};


%%%%%Researcher 1 and Inputs: depends on R_1%%%%%%%
\node [block1, right = 5em of tru](R_2){$R_2$};
\node [block1, above = 8em of R_2](R_1){$R_1$} node [left = -0.1em of R_1, text width = 5em]{Large Treatment Effect};
\node [block1, below = 8em of R_2](R_3){$R_3$} node [left = -0.1em of R_3, text width = 5em]{Small Treatment Effect};

\draw[decoration={brace,mirror}, decorate] (13,-33) -- node[below=6pt, text width = 4.8em](lab1){Researchers Degrees of 
Freedom}(18,-33);

\draw[decoration={brace,mirror}, decorate] (29,-33) -- node[below=6pt, text width = 5em](lab2){Policy Analyst Degrees of 
Freedom}(34,-33);

\draw[decoration={brace}, decorate] (31,33) -- node[below=2pt]{Observed by citizens}(70,33);

%\node [agent](Researcher_1){$R_1$} node [left = 0.6em of Researcher_1, text width = 8em]{Large Treatment Effect};

\node [block1, right = 5em of R_1](PA_12){$PA_{1,2}$};
\node [block1, above = .5em of PA_12](PA_11){$PA_{1,1}$} node [right = -0.1em of PA_11, text width = 5em]{Large gains only};
\node [block1, below = .5em of PA_12](PA_13){$PA_{1,3}$};

\node [block1, right = 5em of R_2](PA_22){$PA_{22,}$};
\node [block1, above = .5em of PA_22](PA_21){$PA_{2,1}$};
\node [block1, below = .5em of PA_22](PA_23){$PA_{2,3}$};

\node [block1, right = 5em of R_3](PA_32){$PA_{3,2}$};
\node [block1, above = .5em of PA_32](PA_31){$PA_{3,1}$};
\node [block1, below = .5em of PA_32](PA_33){$PA_{3,3}$} node [right = -0.1em of PA_33, text width = 5em]{Large losses only};


\node [block2, above right = 3em and 15em of R_2](PM_1){Policy\linebreak Maker 1};
\node [block2, below right = 3em and 15em of R_2](PM_2){Policy\linebreak Maker 2};

\node [block3, right = 3em of PM_1](PC_1){Support};
\node [block3, right = 3em of PM_2](PC_2){Oppose};


%%%%% Draw edges col 2%%%%%%%%%%%%%%%%%
%Truth to research
\draw [line](tru) -- (R_1);
\draw [line_d](tru) -- (R_2);
\draw [line_d](tru) -- (R_3);


%Research  to PA
\draw [line_d](R_1) -- (PA_13);
\draw [line_d](R_1) -- (PA_11);
\draw [line](R_1) -- (PA_12);

\draw [line_d](R_2) -- (PA_23);
\draw [line_d](R_2) -- (PA_21);
\draw [line_d](R_2) -- (PA_22);

\draw [line_d](R_3) -- (PA_33);
\draw [line_d](R_3) -- (PA_31);
\draw [line_d](R_3) -- (PA_32);



%PA to PM
\draw [line](PA_11) -- (PM_1);
\draw [line](PA_33) -- (PM_2);
%PM to PC
\draw [line](PM_1) -- (PC_1);
\draw [line](PM_2) -- (PC_2);

%%%%%%%%%%%%%%%%%%%%%%%%%%%%%%%%%%%%%%




\end{tikzpicture}
\end{figure}
%\end{comment}
\end{frame}


\begin{comment}

\begin{frame}{}
\begin{table}[ht]
\centering
\begin{tabular}[t]{|l|c|c|}
\hline
& Empirical  & Policy \\
& Research & Analysis \\

\hline
Problems & Reproducibility  &  Credibility \\
				 &  Crisis & Crisis \\
\hline
Solutions & {\white  \textit{Open Science } }&    {\white \textit{Open Policy Analysis } }\\
 &   {\white Principles, Guidelines, } &   {\white Principles, Guidelines,}\\
 &  {\white  Applications} &   {\white Applications}\\

\hline
\end{tabular}
\end{table}%
\end{frame}


\begin{frame}[noframenumbering]{}
\begin{table}[ht]
\centering
\begin{tabular}[t]{|l|c|c|}
\hline
& Empirical  & Policy \\
& Research & Analysis \\

\hline
Problems & Reproducibility  &  Credibility \\
				 &  Crisis & Crisis \\
\hline
Solutions &  \textit{Open Science }&    {\white \textit{Open Policy Analysis } }\\
 &   Principles, Guidelines, &   {\white Principles, Guidelines,}\\
 &    Applications  &   {\white Applications}\\

\hline
\end{tabular}
\end{table}%
\end{frame}

\end{comment}

\begin{frame}{Relevance}
Main consequences of policy analysis that lacks openness:
\begin{enumerate}
\item Cherry picking evidence.
\item Challenging to automate and improve systematically recurring reports.
\item Difficulty understanding how research informs policy analysis.
\end{enumerate}
\end{frame}

\begin{frame}{Cherry Picking Evidence}
\pause
\begin{exampleblock}{}
  {\large ``When I was director of the CBO, I was very frustrated when we would write a policy report [saying] a certain policy would have these two advantages and these two disadvantages, and the advocates would quote only the part about the advantages, and the opponents would quote only the part about the disadvantages. That encourages the view that there are simple answers. There aren't generally simple answers. There are trade-offs.''
%I would strayed from Josh Angrist.. what is the right word... [DC says] straight jacket (in a friendly tone). That is a %good one. Although if I were to have a hierarchy of what is the strongest, I would probably start wearing Josh's straight %jacket
}
  \vskip3mm
  \raggedleft{\small--- Douglas Elmendorf (Director of CBO, 2009-2015)} \footnotesize{ \linebreak  Harvard Magazine, 2016} 
  	  
\end{exampleblock}
\end{frame} 

%6mins



\begin{frame}{Difficulty Understanding how Research Informs Policy Analysis}
\begin{itemize}
\item What happens when new research emerges?
\begin{itemize}
\item What if $\hat{\tau}($ Blattman, Fiala, and Martinez 2020$) = \frac{1}{2}\hat{\tau}($ Blattman, Fiala, and Martinez 2013$)$? Or $\tau_{2020} = 2\tau_{2013}$?
\end{itemize}
\pause
\item Where a the largest unknowns in the policy analysis?
\begin{itemize}
\item GiveWell lists at least 100 parameters in its cost-effectiveness analysis. What are the 5/10 most important ones?
\end{itemize}
\pause
\item Where is the marginal piece of research most informative for this analysis?
\begin{itemize}
\item Are the gaps in knowledge for this PA guiding the research agenda?
\end{itemize}
\end{itemize}
\end{frame} 


\section[Solutions]{The Open Science Movement}

\begin{frame}{Open Science}
\begin{table}[ht]
\centering
\begin{tabular}[t]{|l|c|c|}
\hline
& Empirical  & Policy \\
& Research & Analysis \\

\hline
Problems & Reproducibility  &  Credibility \\
				 &  Crisis & Crisis \\
\hline
Solutions &  \textit{Open Science }&    {\white \textit{Open Policy Analysis } }\\
 & Principles, Guidelines,  &   {\white Principles, Guidelines,}\\
 & Applications &   {\white Applications}\\

\hline
\end{tabular}
\end{table}%
\end{frame}


\begin{frame}[noframenumbering]{Open Policy Analysis}
\begin{table}[ht]
\centering
\begin{tabular}[t]{|l|c|c|}
\hline
& Empirical  & Policy \\
& Research & Analysis \\

\hline
Problems & Reproducibility  &  Credibility \\
				 &  Crisis & Crisis \\
\hline
Solutions &  \textit{Open Science }&    \textit{Open Policy Analysis } \\
 & Principles, Guidelines,  &   Principles\\
 & Applications &   \\

\hline
\end{tabular}
\end{table}%
\end{frame}


\begin{frame}[shrink=25, label = pa_comp]{The Process of Policy Analysis}

\tikzstyle{estimate} = [diamond, draw, node distance= 7em,text width = 5em, minimum height=5em, minimum width=5em, align = center]
\tikzstyle{line} = [draw, -stealth]
\tikzstyle{line_enph} = [draw, red, ultra thick]
\tikzstyle{inp} = [draw, rectangle, text centered, minimum height=3em, text width=2em, node distance= 2em]
\tikzstyle{source} = [draw, rectangle, text centered, minimum height=8em, text width=5em, node distance= 10em]
\tikzstyle{model} = [draw, rectangle, text centered, minimum height=8em, text width=15em, node distance= 10em]
%\tikzstyle{outp} = [draw, ellipse, text centered, minimum height=2mm, text width=2em, node distance= 10em]
%\tikzstyle{block} = [draw, rectangle,  text width=8em, text centered, minimum height=55mm, node distance= 5em]


%\begin{adjustbox}{max totalsize={1\textwidth}{.8\textheight},center}

\begin{figure}[h!]\centering 
\hspace*{-2.5em}
\begin{tikzpicture}[thick,scale=0.6, every node/.style={scale=0.6}]
\setbeamercovered{invisible}

%%%%%Nodes: Sources%%%%%%%
\node [source](D_1){$Data$};
\node [source, below = 1em of D_1](Lit){$Research$};
\node [source, below = 1em of Lit](OR){\textit{Guess work}};


\draw[decoration={brace,mirror}, decorate] (-1.2,-10) -- node[below=6pt] {$Sources$}(1.1,-10);

\draw[decoration={brace,mirror}, decorate] (4.8,-11.3) -- node[below=6pt] {$Inputs$}(6.2,-11.3);


%node[below = 1em ](OR) -- node[below = 6em]{asd}(OR)


%\onslide<2-5>\node [source, color=red](Lit){$Research$};

%%%%%Nodes: Inputs%%%%%%%
\node [inp, above right = 1em and 6em of  D_1 ](I_1){$I_1$};
\node [inp, below = 1em of I_1](I_2){$I_2$};
\node [inp, below = 8em of I_2](I_j){$I_j$};
\node [inp, below right = 1em and 6em of OR ](I_last){$I_J$};




\path (I_2) -- node[auto=false, rotate=90, anchor=north, outer sep=-0.5em]{\ldots} (I_j);
\path (I_j) -- node[auto=false, rotate=90, anchor=north, outer sep=-0.5em]{\ldots} (I_last);

%%%%%Paths connecting Sources with Inputs%%%%%%%
\draw [line](D_1.east) -- (I_1.west);
\draw [line](D_1.east) -- (I_2.west);
\draw [line, opacity=1, anchor=center](Lit.east) -- (I_j.west);
\draw [line, opacity=1, anchor=center](OR.east) -- (I_last.west);


\draw [line, opacity=.3][xshift=1em](D_1.east) -- ([yshift=-8 em]I_1.west);
\draw [line, opacity=.3][xshift=1em](D_1.east) -- ([yshift=-10 em]I_1.west);

\draw [line, opacity=.3][xshift=1em](Lit.east) -- ([yshift=7 em]I_j.west);
\draw [line, opacity=.3][xshift=1em](Lit.east) -- ([yshift=-5 em]I_j.west);

\draw [line, opacity=.3][xshift=1em](OR.east) -- ([yshift=3 em]I_last.west);
\draw [line, opacity=.3][xshift=1em](OR.east) -- ([yshift=5 em]I_last.west);


%%%%%Node and paths for model%%%%%%%
\node [model, below right = 2em and 10em of D_1](model){$Model$};
\draw [line](I_1) -| (model);
\draw [line](I_2) -| (model);
\draw [line](I_j) |- (model);
\draw [line](I_last) -| (model);


%%%%%Node and paths for policy estimates%%%%%%%
\node [estimate, right = 5em of model](PE_2){$Policy$ $Estimate_2$};
\node [estimate, above = 5em of PE_2](PE_1){$Policy$ $Estimate_1$};
\node [estimate, below = 5em of PE_2](PE_3){$Policy$ $Estimate_3$};


\draw [line] (model.east) -- (PE_1.south west);
\draw [line] (model.east) -- (PE_2.west);
\draw [line] (model.east) -- (PE_3.north west);


\end{tikzpicture}
\end{figure}

\end{frame}

\begin{frame}{Principles for Open Policy Analysis}
Proposed principles:
\medskip
\begin{enumerate}
\item[{\Large 1}] Computational Reproducibility
\medskip
\item[{\Large 2}]  Analytic Transparency
\medskip
\item[{\Large 3}]  Output Transparency 
\end{enumerate}
\end{frame} 

\begin{frame}{Principle 1: Stop re-inventing the wheel}
\begin{columns}
\begin{column}{0.3\textwidth}
\textbf{Computational Reproducibility}
\begin{itemize}
\item Literate \\ Programming
\item Version control
\item File structure
\item \textbf{Label sources}
\end{itemize}
\end{column}
\begin{column}{0.65\textwidth}  %%<--- here
    \begin{center}
     \includegraphics[width=1\textwidth]{../Images/repro.png}
     \end{center}
\end{column}
\end{columns}
\end{frame}

\begin{frame}{Principle 2: Show your work (readable)}
\begin{columns}
\begin{column}{0.25\textwidth}
   \textbf{Analytic Transparency}
   \begin{itemize}
   \item Open code
   \item Open data
   \item  Report as Dynamic Document    
   \end{itemize}
\end{column}
\begin{column}{0.7\textwidth}  %%<--- here
    \begin{center}
     \includegraphics[width=1\textwidth]{../Images/a_transp.png}
     \end{center}
\end{column}
\end{columns}
\end{frame}

\begin{frame}{Principle 3: Let's all agree on one table/viz}
\begin{columns}
\begin{column}{0.3\textwidth}
 \textbf{Output \\ Transparency}
   \begin{itemize}
   \item Pre-committed output display
   \item Assumptions- output link
   \end{itemize}
\end{column}
\begin{column}{0.7\textwidth}  %%<--- here
    \begin{center}
    \vspace{-0.8em}
     \includegraphics[width=.6\textwidth]{../Images/o_transp.png}
     \end{center}
\end{column}
\end{columns}
\end{frame}

\begin{comment}
\section{Suggestions}

\begin{frame}{Suggestions}
\textbf{Suggestions:}
\begin{enumerate}
\item \textbf{Policy Analysts: Just Post It}. \\
Things are moving in this direction. Play a leading role in a credibility revolution for policy analysis. 
\pause 
\item \textbf{Policy Analysis Organizations: Open by Default} \\
Boost in credibility, lower costs in the long run. Examples: GiveWell, and AEI. 
\pause 
\item \textbf{Government Agencies and Funders: Support Open Policy Analysis}
Examples: Require contracted policy analysis to be fully open. Support training and adoption of new tools (VC and DD). Inject resources for the transition.
\end{enumerate}
\end{frame} 
\end{comment}

\section[Conclusion]{Conclusion}

\begin{frame}{Summing Up: Where We Are}
   \begin{tikzpicture}[remember picture,overlay]
       \node[at=(current page.center), xshift = 0em, yshift = -0.5em] {
         \includegraphics[width=.96\paperwidth]{../Images/traditional_pa.PNG}
       };
   \end{tikzpicture}
\end{frame}

\begin{frame}{Summing Up: Where Should We Go}
   \begin{tikzpicture}[remember picture,overlay]
       \node[at=(current page.center), xshift = 0em, yshift = -0.5em] {
         \includegraphics[width=.96\paperwidth]{../Images/open_pa.PNG}
       };
   \end{tikzpicture}
\end{frame}

%\begin{frame}{Conclusion}
%\begin{itemize}
%\item Draw a parallel between reproducibility crisis in empirical research and a credibility crisis in policy analysis. 
%\item Proposed guiding principles for open policy analysis.
%\item Identify challenges and provide recommendations for different stakeholders.
%\end{itemize}
%\end{frame}


%\begin{frame}{Next Steps} 
%\begin{itemize}
%\item Journal Science was key for research. Need high profile producer of policy analysis to buy in. 
%\item Develop community-endorsed guidelines for OPA (similar to TOP Guidelines for Open Science)
%\item Carry out case studies with policy agencies to fine tune guidelines, and build a collection of examples (Hoces de la Guardia 2017). 
%\end{itemize}
%\end{frame} 


\section{Application}
\begin{comment}
\begin{frame}[noframenumbering]{Application}
\begin{table}[ht]
\centering
\begin{tabular}[t]{|l|c|c|}
\hline
& Empirical  & Policy \\
& Research & Analysis \\

\hline
Problems & Reproducibility  &  Credibility \\
				 &  Crisis & Crisis \\
\hline
Solutions &  \textit{Open Science}&    \textit{Open Policy Analysis} \\
 & Principles, Guidelines,  &   Principles{\white, \textbf{Guidelines,}}\\
 & Applications &   {\white \textbf{Applications}}\\

\hline
\end{tabular}
\end{table}%
\end{frame}


\begin{frame}[noframenumbering]{Motivation: Gap On How to Conduct OPA}
\begin{table}[ht]
\centering
\begin{tabular}[t]{|l|c|c|}
\hline
& Empirical  & Policy \\
& Research & Analysis \\

\hline
Problems & Reproducibility  &  Credibility \\
				 &  Crisis & Crisis \\
\hline
Solutions &  \textit{Open Science}&    \textit{Open Policy Analysis} \\
 & Principles, Guidelines,  &   Principles, Guidelines,\\
 & Applications &   Applications\\

\hline
\end{tabular}
\end{table}%
\end{frame}
 
\end{comment}
% \section{Approach}
%\begin{frame}{Approach}
%\begin{itemize}
%item Identify a case study
%\item Define guidelines
%\item Demonstrate how to achieve highest standards of Open Policy Analysis (OPA)x
%\item Use sensitivity analysis to explore biggest policy unknowns 
%\begin{itemize}
%\item  Surprisingly, academic debate around one specific parameter seems less relevant from policy perspective
%\end{itemize}
%\end{itemize}
%\end{frame} 


\begin{frame}[noframenumbering]
\begin{enumerate}
\item Why we need Open Policy Analysis (Hoces de la Guardia, Grant \& Miguel, 2018)
\bigskip
\item \textbf{Application to policy estimates of the minimum wage. }
\end{enumerate}
\end{frame}


\section[Case Study]{Description of Case Study}

\begin{frame}[label =  desc_cs]{Description of Case Study}
% 4mins
\begin{center}
``The Effects of a Minimum-Wage Increase on Employment and Family Income'' 
Congressional Budget Office (2014)
\end{center}

\textbf{Description:} CBO estimated the effects of a raise in the federal minimum wage from \$7.25/hr to \$10.10/hr. 


\textbf{Main policy estimates:}
\begin{itemize}
\item ~500,000 jobs would be lost.
\item 16.5 million workers would receive a salary increase. 
\item Distributional effects: below poverty line (PL) +\$5billion; between one and three PL +\$12billion; between three and six PL +\$2billion; above six PL -\$17billion
\end{itemize}

\textbf{Key research estimate:} Elasticity of labor demand for teenagers in the labor force. 
\end{frame}

\begin{comment}

\begin{frame}{Reasons for Selecting the Case Study}
    \begin{multicols}{2}
    \begin{itemize}
        \item Scalable
        \item Recurrent
        \item Feasible
        \item Relevant:
    \end{itemize}
    \end{multicols}
\begin{figure}[h!]
\centering
\vspace*{-1em}
\hspace*{-1.7em}
\includegraphics[scale = 0.22]{../Images/min_wage_gtrend}
\vspace*{-0.8em}
\caption{Google Search Intensity of ``Minimum Wage''}
\label{mw_gtrend}
\end{figure}	
\end{frame}

\end{comment}


\begin{frame}{Adapting TOP Guidelines to Policy Analysis}
\vspace{-.1in}
\begin{figure}[h!]
\centering
\hspace*{-3em}
\includegraphics[scale = 0.3]{../Images/TOP_screens}
\caption{Screen shot of TOP Guidelines}
\label{pol_est}
\end{figure}	
\end{frame}

\begin{comment}
\section{Guidelines}



\begin{frame}{Summary of Adapted Guidelines}
 \centering
 \resizebox{25em}{10em}{%
    \begin{tabular}{ P{1.2cm} P{2cm} P{3cm}  P{3cm}  P{3cm} }
     \toprule
     {\black Standard} & {\black Level 0} 	& {\black Level 1} & {\black Level 2} & {\black Level 3} \\
     \midrule      \midrule
   {\black Workflow} & {\gray Policy estimates vaguely described}  & {\gray All the inputs, and their corresponding sources, used in the calculations are listed } & {\gray Lvl 1 + Policy estimates are listed, in same unit if possible} & {\gray Lvl 2 + all the components can be modified with little effort} \\
     \midrule
    {\black Data} & {\gray Report says nothing} & {\gray Clearly stated whether all, some components, or none of the data is available, with instructions for access when possible.} & {\gray Lvl 1 + report and data are in same place} & {\gray Lvl 2 + Report has specific lines of code that call the data and changes in the data produce traceable changes in the report} \\
     \midrule
      {\black Methods \& Code} & {\gray Key assumption are listed} & {\gray Methods are described in prose. Large amount of work is required to reproduce qualitatively similar estimates}  & {\gray Methods and described in prose, with detailed formulas, and code is provided as supplementary material} & {\gray Lvl 2 + All is in the same document where changes in the code affect the output automatically} \\
     \toprule
     \multicolumn{5}{ P{11.7cm} }{  \raggedleft {\black\small{ From TOP guidelines (Nosek et al 2015) v1.0.1}} \small{\hyperlink{no_gray}{\beamerbutton{}}} }
   \end{tabular}}
\end{frame}


\begin{frame}[noframenumbering]{Summary of Adapted Guidelines}
 \centering
 \resizebox{25em}{10em}{%
    \begin{tabular}{ P{1.2cm} P{2cm} P{3cm}  P{3cm}  P{3cm} }
     \toprule
    {\black Standard} & {\red Level 0 } 	&  {\red Level 1 } &  {\red Level 2}  & {\red Level 3 }  \\
     \midrule      \midrule
    {\black Workflow} & {\gray Policy estimates vaguely described}  & {\gray All the inputs, and their corresponding sources, used in the calculations are listed } & {\gray Lvl 1 + Policy estimates are listed, in same unit if possible} & {\gray Lvl 2 + all the components can be modified with little effort} \\
     \midrule
    {\black Data} & {\gray Report says nothing} & {\gray Clearly stated whether all, some components, or none of the data is available, with instructions for access when possible.} & {\gray Lvl 1 + report and data are in same place} & {\gray Lvl 2 + Report has specific lines of code that call the data and changes in the data produce traceable changes in the report} \\
     \midrule
      {\black Methods \& Code} & {\gray Key assumption are listed} & {\gray Methods are described in prose. Large amount of work is required to reproduce qualitatively similar estimates}  & {\gray Methods and described in prose, with detailed formulas, and code is provided as supplementary material} & {\gray Lvl 2 + All is in the same document where changes in the code affect the output automatically} \\
     \toprule
     \multicolumn{5}{ P{11.7cm} }{  \raggedleft {\black\small{ From TOP guidelines (Nosek et al 2015) v1.0.1}} \small{\hyperlink{no_gray}{\beamerbutton{}}} }
   \end{tabular}}
\end{frame}


\begin{frame}[noframenumbering]{Summary of Adapted Guidelines}
 \centering
 \resizebox{25em}{10em}{%
    \begin{tabular}{ P{1.2cm} P{2cm} P{3cm}  P{3cm}  P{3cm} }
     \toprule
     {\red Standard} & {\black Level 0} 	& {\black Level 1} & {\black Level 2} & {\black Level 3} \\
     \midrule      \midrule
    {\red Workflow} & {\gray Policy estimates vaguely described}  & {\gray All the inputs, and their corresponding sources, used in the calculations are listed} & {\gray Lvl 1 + Policy estimates are listed, in same unit if possible} & {\gray Lvl 2 + all the components can be modified with little effort} \\
     \midrule
    {\red Data} & {\gray Report says nothing} & {\gray Clearly stated whether all, some components, or none of the data is available, with instructions for access when possible.} & {\gray Lvl 1 + report and data are in same place} & {\gray Lvl 2 + Report has specific lines of code that call the data and changes in the data produce traceable changes in the report} \\
     \midrule
      {\red Methods \& Code} & {\gray Key assumption are listed} & {\gray Methods are described in prose. Large amount of work is required to reproduce qualitatively similar estimates}  & {\gray Methods and described in prose, with detailed formulas, and code is provided as supplementary material} & {\gray Lvl 2 + All is in the same document where changes in the code affect the output automatically} \\
     \toprule
     \multicolumn{5}{ P{11.7cm} }{  \raggedleft {\black\small{ From TOP guidelines (Nosek et al 2015) v1.0.1}} \small{\hyperlink{no_gray}{\beamerbutton{}}} }
   \end{tabular}}
\end{frame}

\begin{frame}[noframenumbering, label=guidelines_sum]{Summary of Adapted Guidelines}
 \centering
 \resizebox{25em}{10em}{%
    \begin{tabular}{ P{1.2cm} P{2cm} P{3cm}  P{3cm}  P{3cm} }
     \toprule
     Standard & Level 0 	& Level 1 & Level 2 & Level 3 \\
     \midrule      \midrule
   {\black Workflow} & {\gray Policy estimates vaguely described}  & {\gray All the inputs, and their corresponding sources, used in the calculations are listed } & {\gray Lvl 1 + Policy estimates are listed, in same unit if possible} & {\gray Lvl 2 + all the components can be modified with little effort} \\
     \midrule
    {\black Data} & {\gray Report says nothing} & {\gray Clearly stated whether all, some components, or none of the data is available, with instructions for access when possible.} & {\gray Lvl 1 + report and data are in same place} & {\gray Lvl 2 + Report has specific lines of code that call the data and changes in the data produce traceable changes in the report} \\
     \midrule
      {\black Methods \& Code} & {\gray Key assumption are listed} & {\gray Methods are described in prose. Large amount of work is required to reproduce qualitatively similar estimates}  & {\gray Methods and described in prose, with detailed formulas, and code is provided as supplementary material} & {\gray Lvl 2 + All is in the same document where changes in the code affect the output automatically} \\
     \toprule
     \multicolumn{5}{ P{11.7cm} }{  \raggedleft {\red\small{ From TOP guidelines (Nosek et al 2015) v1.0.1}} \small{\hyperlink{no_gray}{\beamerbutton{}}} }
   \end{tabular}}
\end{frame}
\end{comment}

\section{Application}




\begin{frame}[label=demo]{Applying Guidelines to Build an Open Report}
\begin{center}

{\Huge
\texttt{\href{https://rpubs.com/fhoces/dd_cbo_mw}{{\blue\underline{DEMO}}}}
}


\end{center}

%Discuss automation 

\end{frame}

\begin{comment}

\begin{frame}[shrink=25, label=map_cbo]{Map the complete policy analysis}

\tikzstyle{estimate} = [diamond, draw, node distance= 7em,text width = 5em, minimum height=5em, minimum width=5em, align = center]
\tikzstyle{line} = [draw, -stealth]
\tikzstyle{line_enph} = [draw, red, ultra thick]
\tikzstyle{inp} = [draw, rectangle, text centered, minimum height=3em, text width=2em, node distance= 2em]
\tikzstyle{source} = [draw, rectangle, text centered, minimum height=8em, text width=5em, node distance= 10em]
\tikzstyle{model} = [draw, rectangle, text centered, minimum height=8em, text width=15em, node distance= 10em]
%\tikzstyle{outp} = [draw, ellipse, text centered, minimum height=2mm, text width=2em, node distance= 10em]
%\tikzstyle{block} = [draw, rectangle,  text width=8em, text centered, minimum height=55mm, node distance= 5em]


\begin{figure}[h!]\centering \label{pa_components_ex}
\hspace*{-2.5em}
\begin{tikzpicture}[thick,scale=0.6, every node/.style={scale=0.6}]
\setbeamercovered{invisible}

%%%%%Nodes: Sources%%%%%%%
\node [source](D_1){CPS ORG; CPS ASEC; CBO 10-Y projections};
\node [source, below = 1em of D_1](Lit){Labor demand for teenagers; Ripple effects};
\node [source, below = 1em of Lit](OR){Extrap. for adults; Aggregate effects; Dist of losses};


\draw[decoration={brace,mirror}, decorate] (-1.2,-10) -- node[below=6pt] {$Sources$}(1.1,-10);

\draw[decoration={brace,mirror}, decorate] (4.8,-11.3) -- node[below=6pt] {$Inputs$}(6.2,-11.3);


%node[below = 1em ](OR) -- node[below = 6em]{asd}(OR)



%%%%%Nodes: Inputs%%%%%%%
\node [inp, above right = 1em and 6em of  D_1 ](I_1){$dF_{w}$};
\node [inp, below = 1em of I_1](I_2){$g_{N}$};
\node [inp, below = 10em of I_2](I_j){$\eta_{teens}$};
\node [inp, below right = 1em and 6em of OR ](I_last){$F_{ext}$};



\path (I_2) -- node[auto=false, rotate=90, anchor=north, outer sep=-0.5em]{\ldots} (I_j);
\path (I_j) -- node[auto=false, rotate=90, anchor=north, outer sep=-0.5em]{\ldots} (I_last);

%%%%%Paths connecting Sources with Inputs%%%%%%%
\draw [line](D_1.east) -- (I_1.west);
\draw [line](D_1.east) -- (I_2.west);
\draw [line, opacity=1, anchor=center](Lit.east) -- (I_j.west);
\draw [line, opacity=1, anchor=center](OR.east) -- (I_last.west);

\draw [line, opacity=.3][xshift=1em](D_1.east) -- ([yshift=-8 em]I_1.west);
\draw [line, opacity=.3][xshift=1em](D_1.east) -- ([yshift=-10 em]I_1.west);

\draw [line, opacity=.3][xshift=1em](Lit.east) -- ([yshift=7 em]I_j.west);
\draw [line, opacity=.3][xshift=1em](Lit.east) -- ([yshift=-3 em]I_j.west);

\draw [line, opacity=.3][xshift=1em](OR.east) -- ([yshift=3 em]I_last.west);
\draw [line, opacity=.3][xshift=1em](OR.east) -- ([yshift=5 em]I_last.west);


%%%%%Node and paths for model%%%%%%%
\node [model, below right = 2em and 10em of D_1](model){\hyperlink{equations}{Equations}: \ref{N_final} - \ref{last.eq} };
\draw [line](I_1) -| (model);
\draw [line](I_2) -| (model);
\draw [line](I_j) |- (model);
\draw [line](I_last) -| (model);

%%%%%Node and paths for policy estimates%%%%%%%
\node [estimate, right = 5em of model](PE_2){Average wage gain};
\node [estimate, above = 5em of PE_2](PE_1){Average wage loss};
\node [estimate, below = 5em of PE_2](PE_3){Average balance loss};


\draw [line] (model.east) -- (PE_1.south west);
\draw [line] (model.east) -- (PE_2.west);
\draw [line] (model.east) -- (PE_3.north west);


\end{tikzpicture}
\end{figure}
\hyperlink{full_model}{\beamerbutton{}}
\hyperlink{equations}{\beamerbutton{}}
\hyperlink{before}{\beamerbutton{}}
\end{frame}

\begin{frame}{All in One Output 1/3}
\begin{figure}[h!]
\centering
\hspace*{-3em}
\includegraphics[scale = 0.13]{../Images/alt_pe1}
\caption{Gains and losses. Different Units}
\end{figure}	
\end{frame}

\begin{frame}{All in One Output 2/3}
\begin{figure}[h!]
\centering
\hspace*{-3em}
\includegraphics[scale = 0.13]{../Images/alt_pe2}
\caption{Gains and losses. Different Denominator}
\end{figure}	
\end{frame}

\begin{frame}{All in One Output 3/3}
\begin{figure}[h!]
\centering
\hspace*{-3em}
\includegraphics[scale = 0.13]{../Images/policy_est}
\caption{Gains and losses. Same units and denominator}
\end{figure}	
\end{frame}

\end{comment}

\begin{comment}

\begin{frame}{Demo Checklist}
\begin{itemize}
\item One-click reproducible \& machine independent.
\begin{itemize}
\item Go to {\blue\texttt{\href{http://www.github.com/fhoces/DD_MW}{\underline{github.com/fhoces/DD_MW}}}}. And {\blue\texttt{\href{https://github.com/fhoces/DD_MW/archive/master.zip}{\underline{download}}}}.
\item Execute \texttt{dd\_cbo\_mw.Rmd}. (Requires  {\blue\texttt{\href{https://cran.r-project.org/}{\underline{R}}}} and {\blue\texttt{\href{https://www.rstudio.com/products/rstudio/download/\#download}{\underline{RStudio}}}}  installed)
\end{itemize}
\item Readable. Weather you know {\large\texttt{R}} or not.
\begin{itemize}
\item Code chunks and narrative.
\item Every computation is wrapped in small functions.
\end{itemize}
\item Complete record of all modifications
\begin{itemize}
\item Diff between two versions. Think tank {\blue\texttt{\href{https://rpubs.com/fhoces/dd_cbo_mw}{\underline{A}}}}  vs Think tank {\blue\texttt{\href{https://rpubs.com/fhoces/dd_alt_mw}{\underline{B}}}} 	
\end{itemize} 
\item Sensitivity analysis.
\end{itemize}
\end{frame}

\end{comment}

\section{Sensitivity Analysis}

\begin{frame}[noframenumbering]{Sensitivity Analysis: Status Quo}
\begin{figure}[h!]
\centering
\hspace*{-3em}
\includegraphics[scale = 0.17]{../Images/policy_est}
\caption{Default settings {\white $ \eta $} }
\end{figure}	
\end{frame}

\begin{frame}{SA: Change in Elasticity of Labor Demand}
\begin{figure}[h!]
\centering
\hspace*{-3em}
\includegraphics[scale = 0.17]{../Images/policy_est_eta001}
\caption{From $\eta^{teens}_{lit} = - 0.1$ to  $\eta^{teens}_{lit} = - 0.01(\Delta^{-}90\%)$}
\end{figure}	
\end{frame}

\begin{frame}[noframenumbering]{Sensitivity Analysis: Status Quo}
\begin{figure}[h!]
\centering
\hspace*{-3em}
\includegraphics[scale = 0.17]{../Images/policy_est}
\caption{Default settings {\white $ \eta $} }
\end{figure}	
\end{frame}

\begin{frame}{SA: Change in Distribution of Balance Loses}
\begin{figure}[h!]
\centering
\hspace*{-3em}
\includegraphics[scale = 0.17]{../Images/policy_est_bl_204040}
\caption{From $(1PL, 6PL) \sim (1\%, 29\%, 70\%)$ to  $(20\%, 40\%, 40\%)$}
\end{figure}	
\end{frame}


\begin{frame}{Sensitivity Analysis For Multiple Parameters}
\vspace*{-1.2em}
\begin{table}
\caption{$\%\Delta W$ for a $\%\Delta$ in inputs. Two sample policy makers.}
\scalebox{0.7}{
\begin{tabular}{ll|c|c|c|c}
\hline
	   &						 & \multicolumn{4}{|c}{Re-distributional Preferences} \\
	   &						 & \multicolumn{2}{|c|}{Dislikes $(\rho = -0.1)$} & \multicolumn{2}{|c}{Likes $(\rho = 0.1)$} \\ \hline
Source & Input                & $10\% \Delta^{+}$ 	& $10\% \Delta^{-}$ & $10\% \Delta^{+}$ 	& $10\% \Delta^{-}$\\ \hline
\multicolumn{2}{l|}{Data}     &        &             		&  	\\
	& Annual wage growth ($g_{w}$)  	&	-3\% 	& 2\% 		&	-2\% 	& 1\%		\\
	& Annual growth in $N$ 			&	0.8\% 	& -0.9\%	 	&	0.5\% 	& -0.5\%	 \\
\multicolumn{2}{l|}{Research}        &        	&      	 	&	\\
    & $\eta_{teen}$     				&  -4\%    	&  4\%      &  -2\%    	&  2\%	    \\
   	& Ripple Scope $(8.7, 11.5)$     &  37\%    	&  -24\%    &  21\%    &  -14\%	    \\
   	& Ripple Intensity $(50\% \Delta w)$& 5\%     &  -5\%     &  3\%     &  -3\%	   \\
\multicolumn{2}{l|}{Guess Work}      &        	&          	&           \\
   	& Extrapolation factor ($F_{ex}$)&  -3\%    	&  2\%    	&  -1\%    &  1\%	    \\ 
   	& Non compliance ($\alpha_{1}$)  &  -7\%    	&  7\%      &  -4\%    &  4\%	    \\
   	& Substitution factor ($F_{sub}$)&      		&  20\%     &     		&  -8\%	    \\ 
   	& Net benefits					&    -5\%  	&  5\%      &     2\%  	&  -2\%	    \\
   	& Distribution of balance losses  &   \multicolumn{2}{c|}{}&    \multicolumn{2}{c}{} \\
    & Current: $(1\%, 29\%, 70\%)$	& 	\multicolumn{2}{c|}{} &   \multicolumn{2}{c}{}\\
	& 	 $(1\%, 4\%, 95\%)$			&    \multicolumn{2}{c|}{22\%}     &    \multicolumn{2}{c}{13\%} 	    \\	 
  	& 	 $(5\%, 35\%, 60\%)$			&    \multicolumn{2}{c|}{-17\%}     &    \multicolumn{2}{c}{-9\%} 	    \\
  	& 	$1/N$						&    \multicolumn{2}{c|}{-129\%}     &    \multicolumn{2}{c}{-73\%}	   \\
\end{tabular}
}
\end{table}
\end{frame}

\section[Discussion]{Discussion}
% 5mins
\setbeamercovered{transparent}

\begin{frame}{Limitations}
\begin{itemize}
\item There is additional scope for reproducibility.
\item Complete case study requires extensive institutional knowledge.
\item Guidelines need to be build based on consensus of practitioners. 
\end{itemize} 
\end{frame}

\begin{frame}{What lies ahead}
\onslide<1>{
Let's assume this becomes the new status quo. 
\begin{itemize}
\item Costs of producing the next report on effects of minimum wage will be very small. 
\item Every additional effort will imply improvements on the ``state of the art'' report (e. g. $dBL$; $\eta(MW), \alpha_{1}(MW)$)
\item Learning about one parameter (QALYs, DWL) will update estimates \textit{across} reports.  
\item Much easier to have a substantive and normative policy debate. Pilot example: {\blue\texttt{\href{https://fhoces.shinyapps.io/example_min_wage/}{\underline{Shiny App!}}}}.
} 
%\onslide<2>{
%\item Two type of contributions ($\sim$software development):
%	\begin{itemize}
%	\item Short term: Within a given time period the model should be taken as given. Less freedom, but direct impact on the policy analysis. 
%	\item Long term: Structural revisions occur in parallel and are incorporated in future cycles of the analysis.
%	 	\end{itemize}
%}
%\onslide<2>{
%\item Who should work on this:
%\begin{itemize}
%\item Analytic reviewers of report; Research division within agencies; Study Commissions (``MWSC or MSWC?'' %\citep{card2016interview})
%\item Public policy schools. 
%\item Think tanks; Bank of knowledge \citep{clemens2016new}.
%\end{itemize}
%\item Next Steps
%\begin{itemize}
%\item BG: Reproduce meta-analysis and simulate new types of evidence; Improve DD; Incorporate %comments (especially from CBO); Deploy online (open source); Write!
%\item AG: $dBL$; $\eta(MW), \alpha_{1}(MW)$ and others ; \texttt{\href{https://%fhoces.shinyapps.io/example_min_wage/}{Shiny App!}}
%\end{itemize}

%}
\end{itemize}
\end{frame}

\begin{frame}{Your next steps to push OPA forward}
\begin{itemize}
\item Collaborate with BITSS to open up your PA. 
\item Fund OPA: directly or conditionally.
\item Train students/analysts in OPA.
\item Present/showcase your OPA. Pioneers: GiveWell, AEI.  
\item Nominate a PA to be open {\blue\texttt{\href{https://goo.gl/forms/ImKnRb5pJWt5apcJ2}{\underline{here}}}}. 
\end{itemize}
\end{frame}

\begin{frame}{An Aspiration}
   \begin{tikzpicture}[remember picture,overlay]
       \node[at=(current page.center), xshift =0em, yshift = -1em] {
         \includegraphics[width=.68\paperwidth]{../Images/aspiration2.PNG}
       };
   \end{tikzpicture}
\end{frame}

\begin{comment}
\begin{frame}{An Aspiration}
   \begin{tikzpicture}[remember picture,overlay]
       \node[at=(current page.center), xshift =0em, yshift = -1.4em] {
         \includegraphics[width=.98\paperwidth]{../Images/aspiration.PNG}
       };
   \end{tikzpicture}
\end{frame}
\end{comment}

\begin{frame}[noframenumbering]
\begin{center}
\vspace*{6em}
{\LARGE Thank you.\\}
\bigskip
{\small
Pre-prints:\\
{\blue \href{https://osf.io/preprints/bitss/jnyqh}{Why OPA} } \\
{\blue \href{https://osf.io/preprints/bitss/ba7tr/}{OPA Case Study}  } \\
\medskip
Slides at \\
{\blue \href{http://www.github.com/fhoces/CBO2018}{github.com/fhoces/CBO2018}  }
\bigskip \\
\href{mailto:fhoces@berkeley.edu}{fhoces@berkeley.edu}
}
\end{center}
\end{frame}

\begin{comment}
\appendix

\begin{frame}[noframenumbering]
\begin{center}
Back-up slides
\end{center}
\end{frame}

\backupbegin

\backupend
\end{comment}



\end{document}

